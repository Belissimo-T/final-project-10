Im Rahmen des Informatikunterrichts wurde ein Programm zur gleichzeitigen Simulation von mehreren Turmites als Abschlussprojekt der Klasse 10 entwickelt. Das Programm ermöglicht einem, das Verhalten sowie die Interaktion dieser Turmites zu untersuchen. Dabei ist es möglich, die Regeln von Turmites spezifisch anzupassen. Mithilfe der Programmiersprache Python sowie dem GUI-Toolkit Qt5 \cite{qt5} konnten alle Musskriterien eines zuvor ausgearbeiteten Pflichtenheftes \cite{pflichtenheft} erreicht werden. Zur Nutzung wird ein Grundverständis für die Thematik vorausgesetzt.